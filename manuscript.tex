\documentclass[b5j,twocolumn]{ltjsarticle}

\usepackage{luatex85,luatexja}
\usepackage{graphicx, xcolor}
\usepackage{mathtools, amssymb}
\usepackage{textcomp}
\usepackage[unicode]{hyperref}
\usepackage{titlesec, titletoc}
\usepackage{fancyvrb}
\usepackage{fvextra}
\usepackage{listings}

% 欧文組版のマイクロタイポグラフィー(細かな読み易さの調整)を有効化
\usepackage{microtype}

% フォントのエンコード
\usepackage[T1]{fontenc}

% OTF フォント(和文フォント)
\usepackage{luacode}
\usepackage{luatexja-ruby}
\usepackage{luatexja-otf}
%\usepackage[noembed,deluxe,jis2004]{luatexja-preset} % 源ノ角/源ノ明朝フォント
\usepackage[deluxe,jis2004]{luatexja-preset} % 源ノ角/源ノ明朝フォント

% フォント設定(数式)
\usepackage[math-style=ISO,bold-style=ISO]{unicode-math}

% フォント設定(欧文)
% main ... 本文フォント。ローマン体。\textrm で使用
% sans ... ヒゲのないフォント。サンセリフ体(和文で言うゴシック体)。\textsf で使用
% mono ... 固定幅フォント。タイプライタ体。ソースコードのリストに使う。\texttt で使用
% math ... 数式フォント
%
\setmathfont{TeX Gyre Termes Math}
%\setmathfont{TeX Gyre Pagella Math}
\setmainfont[Ligatures=TeX, Scale=0.95]{TeX Gyre Termes}
%\setmainfont[Ligatures=TeX, Scale=0.95]{TeX Gyre Pagella}
\setsansfont[Ligatures=TeX, Scale=0.95]{TeX Gyre Heros}
%\setsansfont[Ligatures=TeX, Scale=0.9]{TeX Gyre Adventor}
\setmonofont[Ligatures=TeX, Scale=1]{TeX Gyre Cursor}

% TeX Gyre フォントと OTF フォントの共存
\renewcommand{\bfdefault}{bx}
\renewcommand{\headfont}{\gtfamily\sffamily\bfseries}

\usepackage{orumin}

\title{バージョン管理システムJujutsu入門}
\author{荻野 雅紀}
\date{}

\begin{document}
\pagestyle{empty}
\maketitle

\section{はじめに}

バージョン管理の放棄は過去の放棄であり,過去の放棄は未来の放棄である。

ソフトウェアのソースコードを念頭に置いたバージョン管理システムの源流はベル研究所で開発されたSCCSまで遡る。
CVSなどの集中型バージョン管理システムはネットワーク上のサーバーに変更履歴を集積する機能を搭載し,
インターネットを経由した国際的な共同作業をより容易なものとした。
履歴をローカルに保持しつつ他ノードと交換できる分散型バージョン管理システムはオフラインでの操作と
オンラインでの共同作業を両立した。
BitKeeperの無料ライセンス終了をきっかけにLinus Torvaldsらによって開発されたGitはそのひとつである。

Gitは強力かつ柔軟な履歴操作機能を提供しており,様々なワークフローに対応する。
他方で,その強力さと柔軟さは時に学習の妨げとなる。

そこで,Game of Trees\footnote{\url{https://gameoftrees.org/}}や
Jujutsu\footnote{\url{https://github.com/martinvonz/jj}},
Sapling\footnote{\url{https://sapling-scm.com/}}など,Gitのリモートリポジトリとの相互運用性を確保しつつも
特定のワークフローを志向した操作体系を提供するバージョン管理システムが誕生した。

本稿はバージョン管理システムJujutsuを紹介する。執筆時点でJujutsuの最新バージョンは0.8.0である。

\section{Jujutsuの特徴}

JujutsuはMartin von Zweigbergk(Google)による趣味のプロジェクトとして誕生し,
現在はGoogle社内の次世代バージョン管理システムを目指して開発が進められている。

既に述べたとおり,JujutsuはGitのリモートリポジトリとの相互運用性を備えている。
そのため,Jujutsuを使い始めるためにプロジェクトの全員がGitからJujutsuへと完全移行する必要はなく,
希望する者が個別にJujutsuを導入することができる。

\end{document}
